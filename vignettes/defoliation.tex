\documentclass[nohyper, justified, svgnames]{tufte-handout}\usepackage[]{graphicx}\usepackage[]{color}
%% maxwidth is the original width if it is less than linewidth
%% otherwise use linewidth (to make sure the graphics do not exceed the margin)
\makeatletter
\def\maxwidth{ %
  \ifdim\Gin@nat@width>\linewidth
    \linewidth
  \else
    \Gin@nat@width
  \fi
}
\makeatother

\definecolor{fgcolor}{rgb}{0.878, 0.933, 0.878}
\newcommand{\hlnum}[1]{\textcolor[rgb]{0,1,1}{#1}}%
\newcommand{\hlstr}[1]{\textcolor[rgb]{0,1,1}{#1}}%
\newcommand{\hlcom}[1]{\textcolor[rgb]{0.314,0.439,0.502}{#1}}%
\newcommand{\hlopt}[1]{\textcolor[rgb]{0.878,0.933,0.878}{#1}}%
\newcommand{\hlstd}[1]{\textcolor[rgb]{0.878,0.933,0.878}{#1}}%
\newcommand{\hlkwa}[1]{\textcolor[rgb]{0.565,0.933,0.565}{#1}}%
\newcommand{\hlkwb}[1]{\textcolor[rgb]{0.678,0.847,0.902}{#1}}%
\newcommand{\hlkwc}[1]{\textcolor[rgb]{0.8,0.6,1}{#1}}%
\newcommand{\hlkwd}[1]{\textcolor[rgb]{1,0.6,0.6}{#1}}%

\usepackage{framed}
\makeatletter
\newenvironment{kframe}{%
 \def\at@end@of@kframe{}%
 \ifinner\ifhmode%
  \def\at@end@of@kframe{\end{minipage}}%
  \begin{minipage}{\columnwidth}%
 \fi\fi%
 \def\FrameCommand##1{\hskip\@totalleftmargin \hskip-\fboxsep
 \colorbox{shadecolor}{##1}\hskip-\fboxsep
     % There is no \\@totalrightmargin, so:
     \hskip-\linewidth \hskip-\@totalleftmargin \hskip\columnwidth}%
 \MakeFramed {\advance\hsize-\width
   \@totalleftmargin\z@ \linewidth\hsize
   \@setminipage}}%
 {\par\unskip\endMakeFramed%
 \at@end@of@kframe}
\makeatother

\definecolor{shadecolor}{rgb}{.97, .97, .97}
\definecolor{messagecolor}{rgb}{0, 0, 0}
\definecolor{warningcolor}{rgb}{1, 0, 1}
\definecolor{errorcolor}{rgb}{1, 0, 0}
\newenvironment{knitrout}{}{} % an empty environment to be redefined in TeX

\usepackage{alltt}
\geometry{
  left=2cm, % left margin
  top=2cm,
  marginparwidth=6cm,
  bottom=2cm
}
\usepackage[brazil]{babel}
\usepackage[utf8]{inputenc}
\usepackage[T1]{fontenc}
\usepackage{booktabs}
\usepackage{url}

\makeatletter

\title{Modelo de Regressão COM-Poisson: \textbf{defoliation}}
\author{Eduardo E. R. Junior \& Walmes M. Zeviani}
\date{}
\newcommand{\noun}[1]{\textsc{#1}}

\renewcommand{\textfraction}{0.05}
\renewcommand{\topfraction}{0.8}
\renewcommand{\bottomfraction}{0.8}
\renewcommand{\floatpagefraction}{0.75}

\usepackage[breaklinks=true,pdfstartview=FitH]{hyperref}
\hypersetup{colorlinks, allcolors=., urlcolor=DarkBlue}

\makeatother
\IfFileExists{upquote.sty}{\usepackage{upquote}}{}
\begin{document}







\maketitle
\begin{abstract}
Esta \textit{vignette}\footnote{código fonte disponível em 
\url{https://gitlab.c3sl.ufpr.br/eerj12/tcc_eduardo}} faz parte do 
trabalho de conclusão de curso em Estatística na Universidade Federal do 
Paraná que objetiva a exploração de modelos de regressão para variáveis 
de contagem, por Eduardo Junior sob orientação de Walmes Zeviani.
\end{abstract}


\section{Conjunto de dados}

O cojunto de dados\footnote{Veja detalhes em: \textbf{Impacto de diferentes níveis de desfolha artificial nos estádios fenológicos do algodoeiro.}  \\
\url{http://www.scielo.mec.pt/pdf/rca/v35n1/v35n1a16.pdf}} contém 125 observações 
provenientes de um experimento em casa de vegetação inteiramente 
casualizado com 5 repetições, cujo plantas de algodão 
(\textit{Gossypium hirsutum}) foram submetidas à níveis de desfolha 
artificial (5 níveis) combinados com o estágio fenológico da planta na aplicação (5 níveis). As variáveis presente no conjunto de dados são:

\begin{margintable}
Visualizando os dados no R.
\begin{knitrout}\scriptsize
\definecolor{shadecolor}{rgb}{0.063, 0.188, 0.251}\color{fgcolor}\begin{kframe}
\begin{alltt}
\hlcom{## git <- paste0(}
\hlcom{##     "http://git.leg.ufpr.br/",}
\hlcom{##     "leg/legTools.git")}
\hlcom{## devtools::install_git(git)}

\hlkwd{library}\hlstd{(legTools);} \hlkwd{data}\hlstd{(defoliation)}

\hlkwd{summary}\hlstd{(defoliation[,} \hlkwd{c}\hlstd{(}\hlnum{1}\hlstd{,} \hlnum{2}\hlstd{)])}
\end{alltt}
\begin{verbatim}
        phenol       defol     
 vegetative:25   Min.   :0.00  
 flower bud:25   1st Qu.:0.25  
 blossom   :25   Median :0.50  
 boll      :25   Mean   :0.50  
 boll open :25   3rd Qu.:0.75  
                 Max.   :1.00  
\end{verbatim}
\begin{alltt}
\hlkwd{summary}\hlstd{(defoliation[,} \hlopt{-}\hlkwd{c}\hlstd{(}\hlnum{1}\hlstd{,} \hlnum{2}\hlstd{)])}
\end{alltt}
\begin{verbatim}
      rept       bolls       
 Min.   :1   Min.   : 2.000  
 1st Qu.:2   1st Qu.: 7.000  
 Median :3   Median : 8.000  
 Mean   :3   Mean   : 7.824  
 3rd Qu.:4   3rd Qu.: 9.000  
 Max.   :5   Max.   :13.000  
\end{verbatim}
\end{kframe}
\end{knitrout}

Encurtando nomes, para facilitar a escrita dos códigos
\begin{knitrout}\scriptsize
\definecolor{shadecolor}{rgb}{0.063, 0.188, 0.251}\color{fgcolor}\begin{kframe}
\begin{alltt}
\hlcom{## Estreitando nomes}
\hlstd{defol} \hlkwb{<-} \hlstd{defoliation}
\hlstd{(}\hlkwd{names}\hlstd{(defol)} \hlkwb{<-} \hlkwd{substr}\hlstd{(}
    \hlkwd{names}\hlstd{(defoliation),} \hlnum{1}\hlstd{,} \hlnum{3}\hlstd{))}
\end{alltt}
\begin{verbatim}
[1] "phe" "def" "rep" "bol"
\end{verbatim}
\end{kframe}
\end{knitrout}

\end{margintable}

\begin{itemize}
  \item \texttt{phenol}: estágio fenológico durante a aplicação da 
  desfolha (\textit{
  vegetative, flower bud, blossom, boll, boll open});
  \item \texttt{defol}: nível de desfolha artificial aplicado (
  0, 0.25, 0.5, 0.75, 1);
  \item \texttt{bolls}: número de capulhos produzidos ao final da ciclo
  cultura
\end{itemize}

Pela Figura \ref{fig:descritivo} notamos a relevância das covariáveis 
do estudo, pois as curvas de suavização são razoavelmente distintas para
cada estágio (\texttt{phenol}) e há uma tendência evidente ao considerar 
a variação do nível de desfolha (\texttt{defol}) aplicado, em (a) e (b).
Além disso, de forma complementar o gráfico em (c) apresenta as médias
e variâncias amostrais calculadas nas repetições de cada tratamento e 
notamos a clara evidência de subdispersão, pois todos os pontos estão 
abaixo da linha 1 pra 1 que representa a equidispersão.

\pagebreak

\begin{fullwidth}
\begin{knitrout}\scriptsize
\definecolor{shadecolor}{rgb}{0.063, 0.188, 0.251}\color{fgcolor}
\includegraphics[width=\maxwidth]{images/descritivo-1} 

\end{knitrout}

\stepcounter{figure}
\smallskip\noindent\small Figura \thefigure:
(a) Número de capulhos produzidos pelo nível de desfolha estratificado 
por estágio da planta. (b) Variâncias em funçãos das médias amostrais 
calculadas com basa nas 5 repetições de cada tratamento
\end{fullwidth}

\section{Modelos propostos}

Nesta seção ajustamos modelos estatísticos com 3 diferentes distribuições marginais, $[Y \mid X]$ associadas, \textit{Poisson}, 
\textit{Gamma Count}\footnote{As análises foram baseadas em \textbf{The 
Gamma-count distribution in the analysis of experimental underdispersed
data} \\
\url{http://www.leg.ufpr.br/~walmes/papercompanions/gammacount2014/papercomp.html}}
e \textit{COM-Poisson}. Para comparação considerou-se o ajuste do modelo
\textit{quasi-Poisson}. Além disso diferentes estruturas para o preditor
linear foram consideradas:

\begin{itemize}
    \item {\tt Modelo 0:} $\beta_0$ (1 parâmetro)
    \item {\tt Modelo 1:} $\beta_0 + \beta_1 def$ (2 parâmetros)
    \item {\tt Modelo 2:} $\beta_0 + \beta_1 def + \beta_2 def^2$
    (3 parâmetros)
    \item {\tt Modelo 3:} $\beta_0 + \beta_{1phe} def + \beta_2 
    def^2$  (7 parâmetros)
    \item {\tt Modelo 4:} $\beta_0 + \beta_{1phe} def + \beta_{2phe}
    def^2$ (11 parâmetros)
\end{itemize}




\begin{knitrout}\scriptsize
\definecolor{shadecolor}{rgb}{0.063, 0.188, 0.251}\color{fgcolor}\begin{kframe}
\begin{alltt}
\hlcom{##----------------------------------------------------------------------}
\hlcom{## Estimação dos modelos}

\hlcom{##-------------------------------------------}
\hlcom{## Modelo Poisson}
\hlstd{cpP0} \hlkwb{<-} \hlkwd{glm}\hlstd{(bol} \hlopt{~} \hlnum{1}\hlstd{,} \hlkwc{data} \hlstd{= defol,} \hlkwc{family} \hlstd{= poisson)}
\hlstd{cpP1} \hlkwb{<-} \hlkwd{glm}\hlstd{(bol} \hlopt{~} \hlstd{def,} \hlkwc{data} \hlstd{= defol,} \hlkwc{family} \hlstd{= poisson)}
\hlstd{cpP2} \hlkwb{<-} \hlkwd{glm}\hlstd{(bol} \hlopt{~} \hlstd{def} \hlopt{+} \hlkwd{I}\hlstd{(def}\hlopt{^}\hlnum{2}\hlstd{),} \hlkwc{data} \hlstd{= defol,} \hlkwc{family} \hlstd{= poisson)}
\hlstd{cpP3} \hlkwb{<-} \hlkwd{glm}\hlstd{(bol} \hlopt{~} \hlstd{phe}\hlopt{:}\hlstd{def} \hlopt{+} \hlkwd{I}\hlstd{(def}\hlopt{^}\hlnum{2}\hlstd{),} \hlkwc{data} \hlstd{= defol,} \hlkwc{family} \hlstd{= poisson)}
\hlstd{cpP4} \hlkwb{<-} \hlkwd{glm}\hlstd{(bol} \hlopt{~} \hlstd{phe}\hlopt{:}\hlstd{(def} \hlopt{+} \hlkwd{I}\hlstd{(def}\hlopt{^}\hlnum{2}\hlstd{)),} \hlkwc{data} \hlstd{= defol,} \hlkwc{family} \hlstd{= poisson)}

\hlcom{##-------------------------------------------}
\hlcom{## Modelo Quase Poisson}
\hlstd{cpQ0} \hlkwb{<-} \hlkwd{glm}\hlstd{(}\hlkwd{formula}\hlstd{(cpP0),} \hlkwc{data} \hlstd{= defol,} \hlkwc{family} \hlstd{= quasipoisson)}
\hlstd{cpQ1} \hlkwb{<-} \hlkwd{glm}\hlstd{(}\hlkwd{formula}\hlstd{(cpP1),} \hlkwc{data} \hlstd{= defol,} \hlkwc{family} \hlstd{= quasipoisson)}
\hlstd{cpQ2} \hlkwb{<-} \hlkwd{glm}\hlstd{(}\hlkwd{formula}\hlstd{(cpP2),} \hlkwc{data} \hlstd{= defol,} \hlkwc{family} \hlstd{= quasipoisson)}
\hlstd{cpQ3} \hlkwb{<-} \hlkwd{glm}\hlstd{(}\hlkwd{formula}\hlstd{(cpP3),} \hlkwc{data} \hlstd{= defol,} \hlkwc{family} \hlstd{= quasipoisson)}
\hlstd{cpQ4} \hlkwb{<-} \hlkwd{glm}\hlstd{(}\hlkwd{formula}\hlstd{(cpP4),} \hlkwc{data} \hlstd{= defol,} \hlkwc{family} \hlstd{= quasipoisson)}

\hlcom{##-------------------------------------------}
\hlcom{## Modelo Contagem Gama}
\hlstd{cpG0} \hlkwb{<-} \hlkwd{poi2cg}\hlstd{(cpP0)}
\hlstd{cpG1} \hlkwb{<-} \hlkwd{poi2cg}\hlstd{(cpP1)}
\hlstd{cpG2} \hlkwb{<-} \hlkwd{poi2cg}\hlstd{(cpP2)}
\hlstd{cpG3} \hlkwb{<-} \hlkwd{poi2cg}\hlstd{(cpP3)}
\hlstd{cpG4} \hlkwb{<-} \hlkwd{poi2cg}\hlstd{(cpP4)}

\hlcom{##-------------------------------------------}
\hlcom{## Modelo COM-poisson}
\hlstd{cpC0} \hlkwb{<-} \hlkwd{glm.comp}\hlstd{(}\hlkwd{formula}\hlstd{(cpP0),} \hlkwc{data} \hlstd{= defol)}
\hlstd{cpC1} \hlkwb{<-} \hlkwd{glm.comp}\hlstd{(}\hlkwd{formula}\hlstd{(cpP1),} \hlkwc{data} \hlstd{= defol)}
\hlstd{cpC2} \hlkwb{<-} \hlkwd{glm.comp}\hlstd{(}\hlkwd{formula}\hlstd{(cpP2),} \hlkwc{data} \hlstd{= defol)}
\hlstd{cpC3} \hlkwb{<-} \hlkwd{glm.comp}\hlstd{(}\hlkwd{formula}\hlstd{(cpP3),} \hlkwc{data} \hlstd{= defol)}
\hlstd{cpC4} \hlkwb{<-} \hlkwd{glm.comp}\hlstd{(}\hlkwd{formula}\hlstd{(cpP4),} \hlkwc{data} \hlstd{= defol)}
\end{alltt}
\end{kframe}
\end{knitrout}



% latex table generated in R 3.2.2 by xtable 1.7-4 package
% Mon Nov  9 14:54:48 2015
\begin{margintable}
\caption{Log-verossimilhanças} 
\label{logliks}
\begin{tabular}{rrr}
  \hline
Modelos & Gamma.Count & COM.Poisson \\ 
  \hline
  0 & -272.40 & -272.48 \\ 
    1 & -257.35 & -257.46 \\ 
    2 & -255.98 & -256.09 \\ 
    3 & -220.15 & -222.59 \\ 
    4 & -208.39 & -214.47 \\ 
   \hline
\end{tabular}
\end{margintable}


\begin{knitrout}\scriptsize
\definecolor{shadecolor}{rgb}{0.063, 0.188, 0.251}\color{fgcolor}\begin{kframe}
\begin{alltt}
\hlkwd{anova}\hlstd{(cpP0, cpP1, cpP2, cpP3, cpP4,} \hlkwc{test} \hlstd{=} \hlstr{"Chisq"}\hlstd{)}
\end{alltt}
\begin{verbatim}
Analysis of Deviance Table

Model 1: bol ~ 1
Model 2: bol ~ def
Model 3: bol ~ def + I(def^2)
Model 4: bol ~ phe:def + I(def^2)
Model 5: bol ~ phe:(def + I(def^2))
  Resid. Df Resid. Dev Df Deviance  Pr(>Chi)    
1       124     75.514                          
2       123     59.650  1  15.8643 6.805e-05 ***
3       122     58.357  1   1.2926    0.2556    
4       118     32.997  4  25.3604 4.258e-05 ***
5       114     27.255  4   5.7420    0.2193    
---
Signif. codes:  0 '***' 0.001 '**' 0.01 '*' 0.05 '.' 0.1 ' ' 1
\end{verbatim}
\begin{alltt}
\hlkwd{anova}\hlstd{(cpQ0, cpQ1, cpQ2, cpQ3, cpQ4,} \hlkwc{test} \hlstd{=} \hlstr{"F"}\hlstd{)}
\end{alltt}
\begin{verbatim}
Analysis of Deviance Table

Model 1: bol ~ 1
Model 2: bol ~ def
Model 3: bol ~ def + I(def^2)
Model 4: bol ~ phe:def + I(def^2)
Model 5: bol ~ phe:(def + I(def^2))
  Resid. Df Resid. Dev Df Deviance       F    Pr(>F)    
1       124     75.514                                  
2       123     59.650  1  15.8643 65.8202 6.343e-13 ***
3       122     58.357  1   1.2926  5.3630 0.0223564 *  
4       118     32.997  4  25.3604 26.3047 1.846e-15 ***
5       114     27.255  4   5.7420  5.9558 0.0002176 ***
---
Signif. codes:  0 '***' 0.001 '**' 0.01 '*' 0.05 '.' 0.1 ' ' 1
\end{verbatim}
\begin{alltt}
\hlkwd{anova.cg}\hlstd{(cpG0, cpG1, cpG2, cpG3, cpG4)}
\end{alltt}
\begin{verbatim}
         ll npar two.ll.dif npar.dif       pvalue
1 -272.3961    2         NA       NA           NA
2 -257.3503    3  30.091519        1 4.121296e-08
3 -255.9811    4   2.738461        1 9.795906e-02
4 -220.1455    8  71.671185        4 1.007088e-14
5 -208.3865   12  23.518023        4 9.975695e-05
\end{verbatim}
\begin{alltt}
\hlkwd{anova.cmp}\hlstd{(cpC0, cpC1, cpC2, cpC3, cpC4)}
\end{alltt}
\begin{verbatim}
         ll npar two.ll.dif npar.dif       pvalue
1 -272.4794    2         NA       NA           NA
2 -257.4636    3  30.031524        1 4.250794e-08
3 -256.0893    4   2.748565        1 9.734174e-02
4 -222.5869    8  67.004730        4 9.726501e-14
5 -214.4700   12  16.233801        4 2.720960e-03
\end{verbatim}
\end{kframe}
\end{knitrout}


\end{document}
