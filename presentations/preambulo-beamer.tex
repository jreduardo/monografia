%% ======================================================================
%% Eduardo Junior
%% eduardo.jr@ufpr.br
%% 22-11-2015
%% ======================================================================
%% Template latex beamer para ser usado na opção `include: in_header:`
%% de arquivos Rmd com ouput beamer_template

%% ======================================================================
%% README

% O YAML do documento Rmd usado é, modificações podem alterar os
% resutados obtidos.
% ---
% date: "`r format(Sys.time(), '%d de %B de %Y')`"
% fontsize: "10pt"
% output: 
%   beamer_presentation: 
%     fig_caption: yes
%     highlight: zenburn
%     includes:
%       in_header: preambulo-beamer.tex
%     keep_tex: yes
%     slide_level: 3
% ---

%% ======================================================================
%% Lista de pacotes para tarefas gerais no documento

\usepackage[brazil]{babel}
\usepackage{xcolor}
\usepackage{graphicx}
\usepackage{multicol}
\usepackage{setspace}

%% ======================================================================
%% Funções matematicas
\usepackage{amsmath,amssymb,amstext}
\usepackage{mathtools}                  %% Funcionalidades (como \dcases)
\usepackage{dsfont}                     %% Para \mathds{1} Indicadora

\DeclareMathOperator{\Ell}{\mathcal{L}}
\DeclareMathOperator{\R}{\mathbb{R}}
\DeclareMathOperator{\ind}{\mathds{1}}

\DeclareRobustCommand{\rchi}{{\mathpalette\irchi\relax}}
\newcommand{\irchi}[2]{\raisebox{\depth}{$#1\chi$}}

%% ======================================================================
%% Para tabelas
\usepackage{colortbl}
\usepackage{booktabs}
\usepackage{multirow}
\usepackage{array}

\definecolor{url}{HTML}{000080}
\definecolor{run}{HTML}{4A0082}

\hypersetup{colorlinks, allcolors=.,
  urlcolor=url, runcolor=run}

%% ======================================================================
%% Tema e cores do documento

\usetheme{CambridgeUS}
\setbeamertemplate{itemize items}[triangle]
\setbeamertemplate{navigation symbols}{}

\setbeamertemplate{frametitle}{
  \nointerlineskip
  \begin{beamercolorbox}[sep=0.3cm, ht=1.8em, 
    wd=\paperwidth]{frametitle}
    \vbox{}\vskip-2ex%
    \strut\hspace*{3ex}\large\bfseries\insertframetitle\strut
    \vskip-0.8ex%
  \end{beamercolorbox}
}

\setbeamercolor{frametitle}{bg=white, fg=teal}
\setbeamercolor{structure}{fg=teal}
\setbeamercolor{palette primary}{bg=gray!30, fg=teal}
\setbeamercolor{palette tertiary}{bg=teal, fg=white}
\setbeamercolor{footlinecolor}{fg=white,bg=teal}
\setbeamercolor{caption name}{fg=teal}

%% ======================================================================
%% Espaçamento entre itens de uma lista

\renewcommand{\tightlist}{%
  \setlength{\itemsep}{5pt}\setlength{\parskip}{0pt}}

%% ======================================================================
%% Página Inicial

\setbeamertemplate{title page}[default]
\setbeamercolor{title}{fg=teal}
\setbeamercolor{author}{fg=black!70}
\setbeamercolor{institute}{fg=black!70}
\setbeamercolor{date}{fg=black!70}
\setbeamerfont{title}{series=\bfseries}

%% ======================================================================
%% Definição do cabeçalho e rodapé

\setbeamerfont{headline}{size=\fontsize{5}{5}\selectfont}
\setbeamertemplate{headline}{\bfseries
  \leavevmode%
  \hbox{%
    \begin{beamercolorbox}[wd=.5\paperwidth, ht=2.2ex, dp=1ex, right, 
      rightskip=1em]{section in head/foot}
      \hspace*{2ex}\insertsectionhead
    \end{beamercolorbox}%
    \begin{beamercolorbox}[wd=.5\paperwidth, ht=2.2ex, dp=1ex, left,
      leftskip=1em]{subsection in head/foot}
      \insertsubsectionhead\hspace*{2ex}
    \end{beamercolorbox}}
  \vskip0pt
}

\setbeamerfont{footline}{size=\fontsize{5}{5}\selectfont}
\makeatletter
\setbeamertemplate{footline}{\ttfamily\bfseries
  \leavevmode%
  \hbox{%
    \begin{beamercolorbox}[wd=.3\paperwidth, ht=2.4ex, dp=1ex, right, 
      rightskip=1em]{footlinecolor}
      \insertshortauthor%
    \end{beamercolorbox}%
    \begin{beamercolorbox}[wd=.6\paperwidth, ht=2.4ex, dp=1ex, left,
      leftskip=1em]{footlinecolor}
      \hfill\insertshorttitle%
    \end{beamercolorbox}%
    \begin{beamercolorbox}[wd=.1\paperwidth, ht=2.4ex, dp=1ex, left,
      leftskip=1em]{footlinecolor}
      Slide \insertframenumber
    \end{beamercolorbox}}
  \vskip0pt
}
\makeatother

%% ======================================================================
%% Fontes

\usepackage{mathpazo}
\usepackage{inconsolata}
\usepackage{verbatim}

\usefonttheme{professionalfonts}
\usefonttheme{serif}
%% \usepackage{palatino}
%% \usepackage{eulervm}
%% \usepackage[none]{ubuntu}
%% \usepackage{verbatim}
%% 
%% \usefonttheme{professionalfonts}
%% \usefonttheme{serif}
%% \renewcommand{\ttdefault}{ubuntumono}
%% \renewcommand{\ttfamily}{\fontUbuntuMono}
%% \makeatletter
%% \def\verbatim@font{\normalsize\fontUbuntuMono}
%% \makeatother

%% ======================================================================
%% Colorindo ambiente verbatim

\usepackage{etoolbox}
\makeatletter
\preto{\@verbatim}{\topsep=1pt \partopsep=1pt }
\makeatother

\let\oldv\verbatim
\let\oldendv\endverbatim

\def\verbatim{\par\setbox0\vbox\bgroup\oldv}
\def\endverbatim{\oldendv\egroup \hspace*{-1ex} \fboxsep2pt \noindent\colorbox[gray]{0.8}{\usebox0}\par}


%% ======================================================================
%% Layout do tableofcontents

\setbeamertemplate{section in toc}{
  {\color{teal} \bfseries\inserttocsectionnumber.}~
  {\color{black}\inserttocsection}
}

%% ======================================================================
%% Formatando slides para seções e subseções

\AtBeginSection[]{
  \begin{frame}[c, allowframebreaks]
    \begin{center}
      \textcolor{teal}{\thesection} \\ \vspace{0.3cm}
      \parbox{0.6\textwidth}{
        \centering \textcolor{teal}{\LARGE \bf \insertsection}}\\
    \end{center}
  \end{frame}
}

\AtBeginSubsection{
  \begin{frame}[c, allowframebreaks]
    \begin{center}
      \textcolor{teal}{\thesection.\thesubsection} \\ \vspace{0.3cm}
      \parbox{0.6\textwidth}{
        \centering \textcolor{teal!70}{\large \insertsection}\\
        \centering \textcolor{teal}{\Large \bf \insertsubsection}}\\
    \end{center}
  \end{frame}
}

%% ======================================================================
%% Metadados não disponíveis no YAML com o atual default.tex template

\title[Extensões e Aplicações do Modelo COM-Poisson]{Extensões e
  Aplicações do Modelo de Regressão Conway-Maxwell-Poisson para
  Modelagem de Dados de Contagem} 
\author[Eduardo E. R. Junior \& Walmes M. Zeviani]{
  Eduardo Elias Ribeiro Junior \\
  Orientação: Prof. Dr. Walmes Marques Zeviani}
\institute[]{
  Trabalho de Conclusão de Curso - Laboratório B\\
  Departamento de Estatística (DEST)\\
  Universidade Federal do Paraná (UFPR)
}

%% ======================================================================
%% TODO

% * Realizar as definidos em um documento template, para ser usado no YAML
% como `template: template.tex`. Assim não são sobrepostas as definição
% padrão do knitr com as modificações feitas. Utilizar como base o arquivo
% em ~/.../rmarkdown/rmd/latex/default.tex

